\begin{frame}
	\frametitle{\textbf{5. Final remarks}}


The tolerance seems like an easy parameter to figure out, but it is not.

In practice, we will often be solving non-linear problems. This means that we need to solve multiple linear problems of the form:

\[
\mathcal{J}(x) \delta x = - R(x)
\]

where $\mathcal{J}(x)$ is a Jacobian matrix, $\delta x$ is a correction vector and $R(x)$ is the residual vector. When we apply newton Method, the residual will decrease (if everythign goes well) quadratically. 

Consequently, the tolerance of our iterative solver should adapt to our location in the non-linear solution process. For the first iteration, you do not necessarily need to solve $\mathcal{J}(x) \delta x = - R(x)$ accurately, wheras for the last ones you do.

The idea behind this principle of the Eisenstat Walker method.

For some fun, see the following \href{https://scicomp.stackexchange.com/questions/33806/understanding-the-eisenstat-walker-method-for-choosing-the-tolerance-of-a-linear}{\color{orange}stack overflow post} 
    
\end{frame}