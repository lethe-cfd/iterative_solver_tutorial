\begin{frame}
	\frametitle{\textbf{1. Context and motivation}}

    \visible<1->{
    \begin{shaded}
    \textbf{Why?}
    \begin{itemize}
        \item Linear solvers play a large role in the Lethe workflow for FEM simulations.
        \item Hard to self-study.
    \end{itemize}
    \end{shaded}}

    \visible<2->{
    \begin{shaded}
    \textbf{What is covered?}
    \begin{itemize}
        \item Establish two canonical problems which lead to sparse linear systems.
        \item Present four iterative solvers in order of complexity:
        \begin{itemize}
        	\item Jacobi (very easy)
        	\item Gauss Seidel (easy)
        	\item Conjugate Gradient (hard)
        	\item GMRES (very hard)
        \end{itemize}
    \end{itemize}
    \end{shaded}}
    % \vspace{0.4cm}
    \visible<3->{
	\begin{shaded}
		\textbf{What is \textbf{not} covered?}
		\begin{itemize}
			\item Preconditioning  and sparse data structure for matrices.
			\item Proofs for convergence and rate of convergence.
		\end{itemize}
		\end{shaded}}

\end{frame}

\begin{frame}
	\frametitle{\textbf{1. Note}}
	
	\visible<1->{
		\begin{shaded}
			\textbf{Material is available on the lethe github repository}
			\begin{itemize}
				\item Today's slides and code are available on \href{https://github.com/lethe-cfd/learning_iterative_methods_for_sparse_systems}{\color{orange} this github repository}
			\end{itemize}
	\end{shaded}}
	
	\visible<2->{
		\begin{shaded}
			\textbf{References on the topic}
			\begin{itemize}
				\item Canonical book by Yousef Saad : Saad, Yousef. Iterative methods for sparse linear systems. Society for Industrial and Applied Mathematics, 2003.
				\item Available for free online \href{https://www-users.cse.umn.edu/~saad/IterMethBook_2ndEd.pdf}{\color{orange} here}
			\end{itemize}
	\end{shaded}}

	
\end{frame}